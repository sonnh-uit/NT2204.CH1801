\section{Hiện thực ứng dụng}
\label{results}

\subsection{Yêu cầu chung của bài toán}
\label{sec4_sub1}
Hệ thống được triển khai là một hệ thống đơn giản, chỉ bao gồm backend và database. Người dùng sử dụng một domain duy nhất truy cập vào ứng dụng. Tùy vị trí của người dùng, ứng dụng được điều hướng đến các server khác nhau. Ứng dụng có sử dụng cân bằng tải. Database có khả năng chịu lỗi cao, thời gian đọc nhanh.

\subsection{Phân tích thiết kế hệ thống}
\label{sec4_sub2}
Với những yêu cầu như đề ra tại \ref{sec4_sub1}, ta nhận định:

\begin{itemize}
    \item Ứng dụng cần một DNS có khả năng phân giải theo khu vực địa lý. Route53 hỗ trợ tính năng này
    \item Ứng dụng có cân bằng tải, Application Load Balancer phù hợp cho trường hợp này. Các loại load balancer khác như network load balancer cũng có thể đảm nhiệm cùng tình năng. Nhưng ở đây ta cần cân bằng tải ở tầng ứng dụng, việc sử dụng tới layer 3 là không cần thiết.
    \item Database trên AWS hầu hết đều cung cấp chức năng replicate. Vì vậy, ta chọn tùy theo tech stack. Ở đây chọn RDS MySQL.
\end{itemize}

Những lựa chọn trên không hẳn là tốt nhất, nhưng được cân nhắc sao cho vừa thể hiện tính phân tán của hệ thống, vừa sử dụng những tài nguyên ít chi phí, đồng thời có khả năng "trình diễn" sâu. Việc sử dụng các dịch vụ server less cũng được cân nhắc, tuy nhiên sẽ trùng đề tài. Khả năng dựng database trên các instance cũng có thể được thực hiện trong trường hợp production để giảm chi phí.

Tổng quan hệ thống trình bày trong hình \ref{fig:sec4-overall-diagram}.

\begin{figure}
    \centering
    \includegraphics[scale=0.2]{section4/overall-diagram.png}
    \caption{Sơ đồ tổng quan ứng dụng}
    \label{fig:sec4-overall-diagram}
\end{figure}


\begin{figure}
    \centering
    \includegraphics[scale=0.6]{section4/geoproximity_rule.png}
    \caption{Mô tả geoproximity rule của Route 53}
    \label{fig:sec4-geoproximity_rule}
\end{figure}

Tại hình ảnh trên, người dùng sử dụng ứng dụng có domain name được quản lý bởi Route 53. Sử dụng geoproximity policy để điều hướng request tới từng region khác nhau. Nếu địa chỉ IP của người dùng nằm trong vùng màu xanh (số 1), các request sẽ được điều hướng tới region \textit{us-east-1}, ngược lại các request sẽ được xử lý tại region \textit{ap-southeast-1} (xem hình \ref{fig:sec4-geoproximity_rule}).

Tại backend, deploy web service nginx và một service Flask app đơn giản. Ứng dụng này trả về các truy xuất vào các database khi được yêu cầu hoặc trả về hostname của server và timestamp ở mặc định. Trước khi request đi vào backend, sử dụng một Application Load Balancer nhằm cân bằng tải. Nó thực hiện heal check tới backend thông qua port 80 và serve port 80 ra ngoài. Từ bên ngoài không gọi thẳng được tới backend thông qua giao thức http.

Tại database, tiến hành deploy một RDS MySQL chính tại \textit{us-east-1} và một RDS Read Replicate tại \textit{ap-southeast-1} (do chi phí ở \textit{us-east-1} rẻ hơn). AWS hỗ trợ chúng ta đồng bộ dữ liệu cho 2 database này với cam kết cao. Các request yêu cầu database tại \textit{us-east-1} sẽ được xử lý trên RDS Master. Trong khi các request đọc database ở \textit{ap-southeast-1} sẽ được đọc trên RDS Slave. Các yêu cầu ghi database tại \textit{ap-southeast-1} được chuyển tiếp tới backend ở \textit{us-east-1} thông qua một private connection (VPC Peering connection)

\subsection{Hiện thực ứng dụng}

Chúng ta tiến hành tạo các tài nguyên như đã trình bày trong \ref{sec4_sub2}.

Tại region \textit{us-east-1}, tiến hành tạo các tài nguyên EC2 và Application Load Balancer.

\begin{figure}
    \centering
    \includegraphics[scale=0.4]{section4/us_ec2.png} \\
    \includegraphics[scale=0.4]{section4/us_applb.png}
    \caption{Các tài nguyên tạo tại region US}
    \label{fig:us-resources}
\end{figure}

Tương tự trong hình \ref{fig:sing-resources} là các tài nguyên ở region Sing.

\begin{figure}
    \centering
    \includegraphics[scale=0.4]{section4/sing_ec2.png} \\
    \includegraphics[scale=0.4]{section4/sing_applb.png}
    \caption{Các tài nguyên tạo tại region Sing}
    \label{fig:sing-resources}
\end{figure}

Đối với database, database có sự khác biệt giữa Primary và Replica (hình \ref{fig:database})

\begin{figure}
    \centering
    \includegraphics[scale=0.4]{section4/us_db.png} \\
    \includegraphics[scale=0.4]{section4/sing_db.png}
    \caption{Database tại hai region US và Sing}
    \label{fig:database}
\end{figure}


Đối với Route53, ta cấu hình domain \textit{nt2004.sonnh.net} theo rule Geoproximity như trình bày tại \ref{sec4_sub2}. Ngoài ra, để thuận tiện testing, cấu hình thêm các domain \textit{sing-lab.sonnh.net} và \textit{us-lab.sonnh.net} như hình \ref{fig:route53}.

\begin{figure}
    \centering
    \includegraphics[scale=0.4]{section4/route53.png}
    \caption{Cấu hình các record trong Route53}
    \label{fig:route53}
\end{figure}

\subsection{Kết quả ứng dụng trong từng ngữ cảnh}

Kết quả, hình \ref{fig:cross_region_access} thể hiện, khi cùng truy cập một domain, với vị trí tại Việt Nam thì ứng dụng của chúng ta được xử lý bởi backend đặt tại Sing, còn khi chuyển vùng bằng VPN sang US thì lại được xử lý bởi backend đặt tại US.

\begin{figure}
    \centering
    \includegraphics[scale=0.3]{section4/vn_accessed.png} \\
    \includegraphics[scale=0.3]{section4/us_accessed.png}
    \caption{Truy cập cùng một domain ở các vị trí khác nhau}
    \label{fig:cross_region_access}
\end{figure}

Tại app domain \textit{sing-lab.sonnh.net}, ta request hai loại thao tác với database khác nhau (action=create\_table và action=select\_records). Kết quả khi request tạo bảng, là yêu cầu cần ghi, nó được xử lý bởi backend đặt tại US (do forward request sang), ngược lại thì được xử lý ở Sing (xem hình \ref{fig:sing-query})

\begin{figure}
    \centering
    \includegraphics[scale=0.8]{section4/sing_query_create_table.png} \\
    \includegraphics[scale=0.8]{section4/sing_query_select.png}
    \caption{Các yêu cầu khác nhau cho app domain tại Sing}
    \label{fig:sing-query}
\end{figure}

Một ví dụ nữa, ta sẽ thêm các records vào database với domain sing, và sau đó query từ domain us. Hình \ref{fig:sing-us-records} cho thấy, record insert vào database dược xử lý tại US, nhưng bởi server web2, trong khi request lấy dữ liệu lại được xử lý bởi server web1. Có sự xuất hiện này là do ta đã sử dụng Application Load Balancer.

\begin{figure}
    \centering
    \includegraphics[scale=0.5]{section4/sing-query-insert-records.png} \\
    \includegraphics[scale=0.45]{section4/us-query-select-records.png}
    \caption{Thực hiện tạo records và sau đó query với hai domain khác nhau}
    \label{fig:sing-us-records}
\end{figure}