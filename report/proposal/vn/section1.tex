\section{Tóm tắt đề tài}
\label{summary}

Trong hàng thập kỷ phát triển của ngành Công nghệ Thông tin, máy tính luôn là một trong những điểm khởi đầu cũng như nút thắt cuối cùng trong sự phát triển của công nghệ. Rào cản về điện tử và vật lý một mặt thúc đẩy, mặt khác cũng kìm hãm những giới hạn tư duy của các nhà khoa học.

Từ khởi đầu với kích thước lớn như tòa nhà, đến thời điểm vi xử lý chỉ còn vài centimet, từ những trung tâm dữ liệu được xây dựng đầu tiên, tới việc mỗi công ty, tập đoàn, trường đại học đều có một trung tâm dữ liệu riêng, với những cách quản trị khác nhau. Những phát triển nhanh chóng về cơ sở hạ tầng tiếp tục thúc đẩy những nghiên cứu mới vượt qua mọi rào cản kỹ thuật, ứng dụng được nhiều thuật toán, giải pháp trên nền tàng máy tính mới, với tốc độ xử lý, khả năng tổng hợp kho dữ liệu lớn cùng đường truyền tốc độ cao. Thông qua internet, người dùng kết nối thiết bị của mình đến các thiết bị ảo hóa có sức mạnh xử lý khổng lồ ở bất kỳ nơi nào trên thế giới \cite{furht2010handbook}.

Hệ thống của các nhà cung cấp điện toán đám mây phân bổ khắp thế giới, sử dụng nhiều thuật toán, chương trình, thiết bị, vật liệu và hàng ngàn kỹ sư để duy trì kết nối với nhau, cung cấp dịch vụ liên tục cho chính hệ thống của họ và thương mại hóa.

Điện toán đám mây là sự tổng hợp của Distributed Computing, Parallel Computing, Utility Computing along with Network Storage, Virtualization, Load Balance, High Available và các công nghệ liên quan khác. Có nhiều định nghĩa khác nhau về Điện toán đám mây, một trong số định nghĩa phổ biến thường được sử đụng đó là: “Điện toán đám mây là một mô hình phổ biến, thuận tiện, cho phép truy cập từ mạng cục bộ tới các tài nguyên tính toán có thể tùy chỉnh cấu hình, được cung cấp và có thể sử dụng nhanh chóng với những tác động tối thiểu từ nhà cung cấp dịch vụ” \cite{mell2011nist}.

Trong khuôn khổ đồ án này, nhóm học viên giới thiệu một số giải pháp được cung cấp của dịch vụ Amazon Web Services, trình bày khả năng của nó trong việc đảm bảo cung cấp dịch vụ liên tục một cách phân tán, hệ thống được cung cấp cho người dùng mà họ không hay biết nó được triển khai trên các thành phần khác nhau. 

Phần còn lại của báo cáo được trình bày như sau:
\begin{itemize}
    \item Phần hai trình bày lịch sử hình thành và một số khái niệm chính của điện toán đám mây nói chung cũng như của AWS nói riêng.
    \item Phần ba trình bày chi tiết các dịch vụ được sử dụng trong báo cáo, các loại hình dịch vụ, những điểm mạnh và các phần còn hạn chế.
    \item Phần bốn trình bày mô hình, ngữ cảnh ứng dụng. Các thành phần hệ thống, phương pháp triển khai và biểu diễn kết quả
    \item Phần năm trình bày kết luận và các nội dung liên quan.
\end{itemize}