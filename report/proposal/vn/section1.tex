\section{Tóm tắt đề tài}
\label{summary}

Khoa học và công nghệ đã và đang có những bước phát triển vĩ đại trong vài thập kỷ vừa qua. Các công trình nghiên cứu bước ra khỏi phòng thí nghiệm và phục vụ đại chúng, trong đó có các ứng dụng cung cấp khả năng tính toán vượt bậc, giá thành rẻ, dễ sử dụng được gọi là điện toán đám mây. Thông qua internet, người dùng kết nối thiết bị của mình đến các thiết bị ảo hóa có sức mạnh xử lý khổng lồ ở bất kỳ nơi nào trên thế giới \cite{furht2010handbook}.

Hệ thống của các nhà cung cấp điện toán đám mây phân bổ khắp thế giới, sử dụng nhiều thuật toán, chương trình, thiết bị, vật liệu và hàng ngàn kỹ sư để duy trì kết nối với nhau, cung cấp dịch vụ liên tục cho chính hệ thống của họ và thương mại hóa.

Trong khuôn khổ đồ án này, nhóm học viên giới thiệu một số giải pháp được cung cấp của dịch vụ Amazon Web Services, trình bày khả năng của nó trong việc đảm bảo cung cấp dịch vụ liên tục một cách phân tán, hệ thống được cung cấp cho người dùng mà họ không hay biết nó được triển khai trên các thành phần khác nhau. 

Phần còn lại của báo cáo được trình bày như sau:
\begin{itemize}
    \item Phần hai trình bày lịch sử hình thành và một số khái niệm chính của điện toán đám mây nói chung cũng như của AWS nói riêng.
    \item Phần ba trình bày chi tiết các dịch vụ được sử dụng trong báo cáo, các loại hình dịch vụ, những điểm mạnh và các phần còn hạn chế.
    \item Phần bốn trình bày mô hình, ngữ cảnh ứng dụng. Các thành phần hệ thống, phương pháp triển khai và biểu diễn kết quả
    \item Phần năm trình bày kết luận và các nội dung liên quan.
\end{itemize}